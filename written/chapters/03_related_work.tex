% !TeX root = ../main.tex

\chapter{Related Work}\label{chapter:related_work}
Lots of research has been done in the area of detecting copy-pasted code.
This mainly results from the huge field of application, where such techniques can be used.
Most of the research can be differentiated in the types of copied code which they detect.
In the case of this work, code which differs in formatting and small added/removed or rearranged statements are of advantage. % TODO why?

\section{Detecting Similarities in Source Code}\label{section:related_work/detecting_similarities}
On of the earlies areas is detection of copy-pasted code in student submissions of programming lessons, which is researched since more than 30 years\cite{lancaster2004comparison}.
String and token based approaches like YAP3\cite{wise1996yap3}, JPLAG\cite{prechelt2002finding} or MOSS\cite{schleimer2003winnowing} use greedy-string-tiling, sometimes in combination with anagrams, to find similarities in files.
Other approaches like dup\cite{baker1995finding}, SID\cite{chen2004shared} or GPLAG\cite{liu2006gplag} build a data structures which represents the structure of the programs.
Those structures are then used to compare the programs.

Student submissions are usually aiming for a solution to a common given problem.
When students copy code for such submissions, they often try to trick the plagiarism detection tools by modifying the copied code.
Because of that, those tools are developed for finding copied code in programs where similarities are expected and therefore concentrate on detecting the potential amount of copy-pasting.
They try to detect clones with heavy modifications like rearranged, added, removed or changed statements.
The purpose of this thesis however, is a little different.
This work concentrates on finding copied code which is easily recognized as such, when seen side-by-side.

\section{Clone Detection}\label{section:related_work/clone_detection}
Another area is the detection of clones in the context of code quality.
Clones are two or more similar sections in code mainly originating from copy-pasting.
% Difference compared to \ref{section:related_work/detecting_similar_code}

One of the tools widely known and used is the one from Kamiya et al. called CCFinder\cite{kamiya2002ccfinder} which features a suffix-tree based approach.
They first transform the code by modifying language dependent syntax (e.g. remove accessibility keywords in Java).
After that, they divide the code into token sequences.
Those tokens are further transformed by replacing variable names and literals.
Now the suffix-tree is built over the token sequences. 
With that it is possible to easily find similar token sequences by traversing the tree down to a leave, which points to a location in a file.

Hummel et al.\cite{hummel2010index} developed an approach, where chunks of code are hashed.
They first tokenize the code and normalize it by removing e.g. comments and unifying variable names.
Now the tokens are split into statements, several of them are grouped into chunks and hashed.
When the hash value is calculated the same way for another section of code, it can be used to look up similar code sequences.

Koschke et al. \cite{koschke2014large,koschke2012large} tried to find license infringements in code.
Using a suffix-tree, they showed that this approach is faster than an index based one.
However, \glqq evaluation shows that [their] approach is faster than index-based techniques when the analysis is run only once. If the analysis is to be conducted multiple times, creating an index pays off\grqq \cite{koschke2014large}.

Both, using a suffix-tree or a index for finding copied code, may be suiting the requirements of this thesis.
However, a suffix-tree is quite difficult to persist on disk, which is a requirement for offline use.
With a huge codebase of billion lines of code, this may become a problem.

%TODO SourcererCC & Cloneworks

\section{Code Search}\label{section:related_work/code_search}
With the growth of the Internet, recent research areas start to aim for search engines for code.
Search engines like Google are good for string based searches, but bad for searching program structure.

Bajracharya et al. tried to fill the gap and developed a search engine for open source code.
They extract additional information from the code such as entities (package, interface, class, declarations, ...), relations (dependencies between classes, files, ...), keywords and fingerprints (structures, types, patterns, ...).
This information is then used to query the database and find code which matches the specified properties.

Keivanloo et al. tried to develop a linked-data repository called SeCold \cite{keivanloo2012leveraging,keivanloo2011internet,keivanloo2011seclone,keivanloo2010semantic}.
The system has different parts which enable a user to search for code clones, similar bytecode or program structure.
The most interesting part of the system is the clone search called SeClone, which is described in \cite{keivanloo2011internet,keivanloo2011seclone}.
For preprocessing they first tokenize the code using ASTs (Abstract Syntax Tree) and a normalization approach similar to CCFinder as described in \cite{kamiya2002ccfinder}.
For indexing, the tokens are feed into two different databases, one for code patterns and one to store type usage which is for reducing the amount of false positives.
To enable detecting clones with added, changed or removed statements, they use a multilevel indexing approach, where lines are hashed, the emerging hashes sorted and after that, hashes for multiple lines are hashed again.
This multilevel approach allows rearranging of statements still resulting in the same hash for the code section.

Although those approaches fit the requirement of collecting huge amounts of code in a database and making it searchable, they aim for finding all code by its structure.
In this work, the goal is to find code which is copy-pasted and only modified slightly, if at all.

\section{Commercially Available Tools}
There are several commercially available tools, namely Blackduck and Flexnet Code Insight.
Both are advertised as able to detect code of freely available open source software (FOSS) in a given codebase.
This enables the user to be notified of possible vulnerabilities in the used code.
According to their website, Blackduck features \glqq the world’s most complete, current and accurate repository and database of open source software\grqq\cite{blackduck}.
However, they do not provide information about it's architecture or used techniques.

%\section{Extracting Relevant Information from the Code}
%A comparison of code can be abstracted to similar information relevant for use case found in both code segments: information has to be extracted relevant for license infringement: exact matches, with small modifications
