% !TeX root = ../main.tex

\chapter{Introduction}\label{chapter:introduction}
blabla motivation

\section{Problem Statement}\label{section:introduction/problem}
%TODO See Title of work!!!
% digital content -> easily copied

To fight these problems, in this work a architecture for tool should be developed, which allows to search a codebase for copied code from open source software available on the Internet.

Huge amounts of code -> Index maintenance high effort -> Can not be done on every machine, where tool is used -> Client-Server
Client-Server: Many requests, high load, high traffic, has to be quick (continuous) -> Index on Client
Index on Client -> bad for updates, huge index -> Partial Index (Bloomfilter, Subhash Table)

\begin{itemize}
	\item Prevent license infringement
	\item Link lib instead of copy is the better approach. This also ensures that the library can be exchanged and updated if vulnerabilities are discovered in its code.
	\item This brings the next point: Vulerabilities in the copied code
\end{itemize}

\section{Contribution}
\subsection*{Research Questions}\label{section:introduction/research_questions}
Many of the works described in \autoref{chapter:related_work} concentrate on finding clones between codebases of software projects.
Only few of them do this in a large scale manner or in regard to license infringements.
This work tries to fill the gap and concentrates on the development of a tool for continuous detection of license infringements and tries to answer the research questions, which are described in the following.

\subsection*{How much resources are needed to create and maintain an index for huge amounts of source code including its history?}
%TODO mehr auf viele clients und viel nachfrage konzentrieren
Huge amounts of source code have been analyzed before many times, see \ref{section:related_work/clone_detection}.
Most of the works are looking on 
 in the magnitude of billion lines of code 
 with finite time and resources

\subsection*{What assertions can be made from matches between a code base and the index?}
\subsection*{How robust does the index have to be in regard to changes in code?}

%TODO ref Ausschreibung


\section{Outline}
% TODO outline of thesis: see sourcererCC paper S.2
The paper is organized as follows.
\autoref{chapter:requirements}