% !TeX root = ../main.tex

\chapter{Introduction}\label{chapter:introduction}
blabla motivation

To fight these problems, in this work a architecture for tool should be developed, which allows to search a codebase for copied code from open source software available on the Internet.
The approach of this work is a server system which holds a search engine for code.

Client server, because of updates and huge database. Index can not be downloaded for every update

%TODO See Title of work!!!

\section{Use cases}\label{section:introduction/use_cases}
As mentioned above, detection of license infringement is one use case for the tool.
Beside that, the tool can also be used to...
\begin{itemize}
	\item Prevent license infringement
	\item Link lib instead of copy is the better approach. This also ensures that the library can be exchanged and updated if vulnerabilities are discovered in its code.
	\item This brings the next point: Vulerabilities in the copied code
\end{itemize}

\section{Research Questions}\label{section:introduction/research_questions}
This work should answer the following research questions:
Is it possible to index huge amounts of sourcecode equivalent ot the size of all open source systems available on the Internet?
How big would such an index be relative to the amount of scanned code?
When, scanning a codebase, in which detail does the history has to be scanned as well? %Commit-based? Tag based?
Is it possible to 
%TODO ref Ausschreibung

% digital content -> easily copied

% TODO content of thesis:
First, the current state of research is reflected.