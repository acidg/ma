% !TeX root = ../main.tex

\chapter{Introduction}\label{chapter:introduction}
In software development, reuse of code can reduce time, effort and even enhance quality \cite{krueger1992software}.
Repositories published on the Internet are a popular source for code which offers the desired functionality.
However, when code is copied from open source software systems, licenses may be violated if the terms and conditions are not met.
So called copy-Left licenses like different versions of the GNU General Public License (GPL), can easily cause violations, especially in closed-source systems, since they demand derivative work to be distributed under the same license.
Software may have to be published as open source afterwards, like in the case of Microsoft's ``Windows USB/DVD Download Tool'' \ \cite{microsoft2009download}.
Even permissive licenses like Apache or MIT require authors, which reuse the software to attribute its origin.
Furthermore, keeping track of reused software is difficult, especially when this is not done from the beginning of a software project.

Finding copied code, which potentially causes licensing issues, is difficult.
Simply comparing the hash of files or reading the file's header is not sufficient to find copied code, since it may be modified, or only parts of a file may be copied.
Instead, code has to be compared in more detail.
The approach presented in this work tries to do so by using techniques known from the area of clone detection.
There are many tools available for finding copied code within and between codebases.
Existing clone detection tools try to find similar code, which can be refactored for reuse to reduce redundancy and enhance software quality.
Often, they are designed for single runs and need a pool of projects which they can use for comparison.
The possibility to use them for detection of license infringements on a continuous basis is very limited because of several issues.
% huge dataset
The data structure for comparison has to be built from a huge dataset of open source projects to find copied code, which is a high effort.
% no history analysis
Often, the history of the projects is not considered which may prevent detection of copied code from older versions of a system, when it was changed in the meantime.
% confidentiality
Uploading the code to a server for analysis often is not an option for confidentiality reasons.

The solution developed in this work, is a tool which allows searching a codebase for copied code from open source software available on the Internet.
To decide whether a given fragment of code is part of open source code, the detection tool indexes thousands of open source systems available on the Internet and maintains a data structure, which makes it possible to quickly search for similar code in the indexed systems.
The proposed approach is based on a client server architecture, where the server keeps track of open source projects including their history.
The server provides an interface to query for code sequences in order to find similar matches in open source systems.
This can be done in a continuous manner ( i.e. for every commit) to track whether incongruously licensed code has been copied to the codebase.
One key feature of the tool is the used filtering technique, which significantly reduces the server's load and speeds up the search process.
A file representing a Bloom filter can be downloaded from the server, which allows clients to filter a large part of code sequences which are not tracked by the server.
The remaining code sequences are then used to query the server which returns the actual matches and further details.

The amount of code which can be tracked by the server is in the range of several billion lines of code including history.
The analysis of existing open source code bases is fast and scalable and can be updated periodically.
To verify that, a prototype of the tool has been implemented and more than 500 million lines of code from approximately 2.000 open source projects were indexed in about 15 hours on a consumer laptop.
The resulting database is about 37 GB and the corresponding Bloom filter has a size of about 200 MB for this amount of code.
The approach not only allows to find licensing issues, but also can be used to monitor the use of open source code in a codebase.
With that information, used libraries can be linked instead of copied and vulnerabilities caused by outdated software can be prevented.

The work is organized as follows.
First, definitions used throughout the thesis as well as requirements are listed in \autoref{chapter:preliminaries}.
Subsequently, related work is depicted in \autoref{chapter:related_work} and based on that, an overview of the approach used for the tool is presented in \autoref{chapter:approach}.
Details of the implementation, decisions and its complications are shown in \autoref{chapter:implementation}.
In \autoref{chapter:evaluation} the prototype is used to analyze several thousand projects, findings are categorized and their validity evaluated.
This leads to a discussion about proposals for future work and the usefulness of the approach in \autoref{chapter:conclusion} and \autoref{chapter:future_work}.