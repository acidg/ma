% !TeX root = ../main.tex

\chapter{Evaluation}\label{chapter:evaluation}
To verify the capability of the approach proposed in this work, the prototypical implementation was tested.
The idea was to find out, whether the approach was capable of detecting copied code in a codebase and, by that, unveil license infringements.
For that purpose, several thousand GPL licensed projects where indexed using the prototypical implementation of the tool.
After that, other open source projects, which are licensed not compatible with GPL and therefore should not contain any GPL licensed code indexed before, where analyzed and emerging matches categorized.

This chapter describes the test setup evaluates timings, database sizes and findings.
Eventually, suggestions to improve the approach are listed.

\section{Test Setup}
As a source for the projects to index, GitHub was chosen.
GitHub's REST API was used to generate a list of 1.000 GPL licensed Java and 1.000 GPL licensed C or C++ projects.
All of the projects were checked out locally resulting in sizes as shown in \autoref{table:locs}.

Indexing those 2.000 projects including their history took 16h on a consumer laptop with 16 GB of RAM and an Intel i7 2.6 GHz CPU with 4 cores and hyper-threading.
The indexing was done independently for Java and C/C++ generating two databases.
The resulting database sizes are 4 GB for Java and 33 GB for C/C++, the bloom filter's sizes are 55 MB and 147 MB for Java and C/C++.

Inspection of several indexed projects showed, the huge difference fo the sizes can be justified in the amount of actual code in the repositories.
It seems like Java projects contain more images, binary files or other resources, compared to the C/C++ projects.

\begin{table}[ht]
	\centering
	\begin{tabular}{l|rrrrr}
		& \textbf{Scanned Files} & \textbf{Lines of Code} & \textbf{Total size} & \textbf{Hashes} & \textbf{Chunks} \\ 
		\hline 
		Java & 836.555 & 57.834.764 & 94 GB & 23.773.065 & 35.069.633 \\
		C/C++ & 1.023.092 & 380.338.327 & 102 GB & 61.349.163 & 221.315.454 \\ 
	\end{tabular}
	\caption{Sizes of the 2.000 indexed projects}\label{table:locs}
\end{table}

After two weeks, the index was updated as described in \ref{section:implementation/history_analysis/update}.
About one third of the reference project contained one or more new tags which were indexed.
The update took one hour with generation of the bloom filter.
This shows that incremental updates of the index are worth the effort of implementation compared to re-indexing the reference projects from scratch.

\section{Findings}
Since the bloom filter can return false positives, matches are expected even if no code has been copied from a reference project into the target project for large enough projects.
Those matches - called false positive filter matches (FPFM) in the remainder of this work - are expected to occur once for roughly every 10.000 statements in the target project, since the probability for a false positive of the bloom filter is calculated to be 0,01\%.

\subsection{Unfiltered Findings}
The results of the test are as shown in \autoref{table:unfiltered_findings}.

\begin{table}[ht]
	\centering
	\begin{tabular}{l|rrr}
		& \textbf{True Positives} & \textbf{False Positives} & \textbf{} \\ 
		\hline 
		Java & 836.555 & 158 (0,01\%) & 94GB \\ 
		Java & 836.555 & 158 (0,01\%) & 94GB \\ 
		Java & 836.555 & 158 (0,01\%) & 94GB \\ 
		Java & 836.555 & 158 (0,01\%) & 94GB \\ 
		Java & 836.555 & 158 (0,01\%) & 94GB \\ 
		\hline
		C/C++ & 1.023.092 & 158 (0,01\%) & 102GB \\ 
		C/C++ & 1.023.092 & 158 (0,01\%) & 102GB \\ 
		C/C++ & 1.023.092 & 158 (0,01\%) & 102GB \\ 
		C/C++ & 1.023.092 & 158 (0,01\%) & 102GB \\ 
		C/C++ & 1.023.092 & 158 (0,01\%) & 102GB \\ 
	\end{tabular}
	\caption{Unfiltered Findings}\label{table:unfiltered_findings}
\end{table} 

Consequential, the results have to be filtered, in order to remove false-positives.

\subsection{Filtered Findings}
To get a better view on the results, they are grouped into four categories:
\begin{description}
	\item [Accidential Clones (AC)]
		Matches which are similar by \glqq accident\grqq, e.g. generated code, interface implementations, declarations of structs/enums, switch-case blocks, table data especially present in C/C++.
	\item[Otherwise Licensed Code (OLC)]
		Code which is actually copied, sometimes from a common source and therefore is no direct license infringement.
		This can be relevant nonetheless, since attribution has to be done correctly, see later this section. %TODO + reverse copy
	\item[License Infringement (LI)]
		Code which has been copied and is violating the license of the reference system.
\end{description}

\begin{itemize}
	\item TODO grouping and aggregation
	\item TODO filter low matchcounts per path
	\item TODO table with match categories
	\item TODO list special cases: reverse copied code
\end{itemize}


\section{Detection of Copied Code} %Detected Clone = License Infringement?
\subsection{Detection of License Infringements} %chunksize? => realtion to size of database
\begin{itemize}
	\item TODO probability estimation: When multiple chunks of same file path (even different versions) => higher probability, when mulitple hits/match => higher probability, when many different files, only small amount of hits/match = low probability
	\item TODO Studie, um thresholds zu ermitteln?
	\item TODO Aussagen wenn match gefunden wurde?
	\item TODO Apache code in GPL: Interface implementierung quite similar: Java list implementation, equals code 
	\item TODO: generated code has to be excluded
\end{itemize}

\subsection{Left out Clones}
\begin{itemize}
	\item TODO Is scanning tags enough?
	\item TODO @Override, blocks with no braces, other limitations by normalization...
\end{itemize}

\section{Performance}
\subsection{Size of the Database}
The space $m$ in bits required to store $n$ values with the optimal number of hash functions can be estimated by 
\begin{equation}\label{equ:bloom_probability} %TODO quelle?
m = -\frac{n\cdot \ln(p)}{(\ln2)^{2}}
\end{equation}
with $p$ being the probability for a false positive.
For one billion hashes and a probability of 0.01\%, which roughly means one false positive for every 10000 lines of code, about 2,4 GB are needed.
A simple hash table would need at least 16 GB for the same amount of data.

\section{Confidentiality - Gathering Information about Searched Code}
\begin{itemize}
	\item TODO What information can be extracted nontheless?
	\item TODO previous/further improve confidentiality: parital hashes
\end{itemize}