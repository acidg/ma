% !TeX root = ../main.tex

\chapter{Conclusion}\label{chapter:conclusion}
The tool developed in this work allows clients to detect code, which has been potentially copied from open source systems and may cause license infringements.
It consists of a server, which is analyzing several thousand open source projects including their history and generates an index.
The server provides an interface which allows clients to query for similar code by comparing hashes of normalized chunks of code.
A Bloom filter, which is generated by the server can be downloaded onto a client and used to decide whether a chunk of code is part of the index.
This not only speeds up the search for copied code on the client, but also reduces the load on the server.
The server returns location details for every queried chunk of code, which is part of the index. 
The client filters and aggregates those results and creates potential matches between files in its codebase and files of open source systems available on the Internet.
Inspection of those matches can then reveal code, which has been copied from open source systems and by that, possible licensing issues can be unveiled.

The evaluation conducted in \autoref{chapter:evaluation} show that the approach is capable of finding code which has been copied from open source systems.
This may help software developers to prevent license violations and see, which open source systems are part of a codebase.
The prototypical implementation was able to index more than 500 million lines of open source software including its history.
This was done on a consumer laptop and took about 16h.
The approach is scalable and can theoretically even track most of the open source systems available on established code sharing platforms like GitHub or BitBucket.
Using a Bloom filter to reduce the number of required requests can speed up the analysis and reduce the load on the server to a fraction.
Evaluation showed that in 25\% of the analyzed projects, issues with licensing were found.
Many developers do not know how and which code can be copied without causing license infringements or just do not care about violating open source licenses.

Although the approach sounds promising, there are some major problems regarding the detection of licensing issues.
Due to the difficulty of finding the original origin of copied code, discovering the actual license is very hard, if possible at all.
Also, there are many licenses from which open source developers can choose.
Owners of open source software can even create a new license if existing ones are not suiting their project.
It is even possible to establish a special agreement between the copyright owners of source code which adds even more confusion to the license-jungle.
All these issues and special cases restrain the approach from automatically determining the concrete license situation and deciding whether a violation is present.

Beside that, the tool developed in the course of this thesis can not only be used to find licensing issues, but also enables developers to monitor used libraries and third party code.
This information can be used to link libraries instead of copying the code.
In the long run, this can improve software quality, since maintaining third party code can be simplified and risks caused by outdated copied code containing vulnerabilities can be reduced.