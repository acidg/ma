\chapter{\abstractname}
Open source code shared on platforms like GitHub or BitBucket, is often licensed for modification and reuse even in commercial systems.
Permissive licenses usually demand to inform about the source of the code, whereas more strict ones like the GNU Public License (GPL) or other copy-left licenses require developers to distribute software which relies on the code as open source under the same license or terms.
When copy-left code is copied into a close-sourced code base, the license scheme is violated and the company owning the codebase may face huge fines, when the violation is revealed.

In this thesis a tool architecture for detecting code which has been copied from open source systems is developed.
The client-server architecture proposed in this thesis uses mechanisms known from clone-detection to create an index, which contains huge amounts of freely available open source code on the server-side.
A bloom filter can be downloaded from the server and used to increase the speed of the search process.
To increase the accuracy, the history of the indexed projects is also taken into account.

The approach is prototypically implemented and evaluated by indexing 2.000 open source projects and analyzing 20 projects for copied code resulting in a database size of 37 GB and a filter of less than 200 MB.