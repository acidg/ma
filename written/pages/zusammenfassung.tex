\chapter{Zusammenfassung}
Open-Source-Software, welche auf Plattformen wie GitHub oder BitBucket veröffentlicht wird, ist oft so lizenziert, dass sie sogar in kommerziellen Systemen wiederverwendet werden kann.
Freizügige Lizenzen fordern im Fall einer Wiederverwendung nur eine Quellenangabe, wohingegen strengere Lizenzen wie die GNU General Public License (GPL) oder andere Copyleft Lizenzen verlangen, dass darauf aufbauender Code unter derselben Lizenz veröffentlicht werden muss.
Wird dennoch lizenzierter Quellcode ohne Quellenangabe verwendet oder in inkompatibel lizenzierte Softwaresysteme übernommen, wird entsprechend die Lizenz des Originalcodes verletzt.
Die Autoren müssen dann ihre Software entsprechend kennzeichnen oder gegebenenfalls sogar als Open Source veröffentlichen.

In dieser Arbeit wird ein Werkzeug entwickelt, welches aus Open-Source-Systemen kopierten Quellcode in einer Codebasis erkennen kann.
Dabei indexiert der Server eine sehr große Anzahl an Open-Source-Systemen und deren Historie.
Er stellt eine Schnittstelle zur Verfügung, welche es ermöglicht, nach Quelltextpassagen mit sehr hoher Ähnlichkeit zu suchen.
Der Server berechnet außerdem eine Filterstruktur, welche von Clients heruntergeladen und benutzt werden kann, um den Suchvorgang enorm zu beschleunigen.
Dies vermindert zudem die Auslastung des Servers.

Der Ansatz wurde prototypisch umgesetzt und bezüglich seiner Performanz untersucht.
Dafür wurden 2.000 Open-Source-Projekte in zwei verschiedenen Sprachen indexiert und anschließend mithilfe des Index 10 weitere Projekte jeder Sprache auf kopierten Quellcode untersucht.
Die erzeugte Datenbank ist ca. 37 GB groß, die Größe des Filters beträgt etwa 200 MB.
Durch manuelle Inspektion wurden in 25\% der analysierten Projekte Lizenzprobleme entdeckt.
Eine Automatisierung des Prozesses ist jedoch sehr kompliziert, da es sehr schwierig ist, die Lizenz einer Datei richtig zu ermitteln.