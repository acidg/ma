\chapter{Zusammenfassung}
Open Source Software, welche auf Plattformen wie GitHub oder Bitbucket veröffentlicht wird, ist oft so lizenziert, dass er sogar in kommerziellen Systemen wiederverwendet werden kann.
Freizügigere Lizenzen fordern im Fall einer Wiederverwendung nur eine Quellenangabe, wohingegen strengere Lizenzen wie die GNU Public License (GPL) oder andere Copy-Left Lizenzen verlangen, dass darauf aufbauender Code unter der selben Lizenz veröffentlicht werden muss.
Wird dennoch so lizenzierter Quellcode in unfreien Softwaresysteme übernommen, wird die Lizenz des Originalcodes verletzt und die Autoren müssen die unfreie Software gegebenenfalls als Open Source veröffentlichen.

In dieser Arbeit wird ein Tool entwickelt, welches aus Open Source Systemen kopierten Quellcode in einer Codebasis erkennen kann.
Dabei indiziert der Server eine riesige Anzahl an Open Source Systemen und deren Historie.
Er stellt eine Schnittstelle zur Verfügung, welcher es ermöglicht nach Quelltextpassagen mit sehr hoher Ähnlichkeit zu suchen.
Der Server berechnet außerdem eine Filterstruktur, welche von Clients heruntergeladen und zum Filtern benutzt werden kann.
Dies beschleunigt den Suchvorgang und vermindert die Auslastung des Servers enorm.

Der Ansatz wurde prototypisch umgesetzt und bezüglich seiner Umsetzbarkeit und Performanz untersucht.
Dafür wurden 2.000 Open Source Projekte in zwei verschiedenen Sprachen indiziert und anschließend mithilfe des Index 10 weitere Projekte jeder Sprache auf kopierten Quellcode untersucht.
Die erzeugte Datenbank ist ca. 37 GB groß, die Größe des Filters beträgt etwa 200 MB.
Durch manuelle Inspektion wurden in 25\% der analysierten Projekte Lizenzprobleme entdeckt.
Eine Automatisierung des Prozesses ist jedoch sehr kompliziert, da es sehr schwierig ist die Lizenz einer Datei richtig zu ermitteln.